\documentclass{mc2015}

%%%%%%%%%%%%%%%%%%%%%%%%%%%%%%%%%%%%%%%%%%%%%%%%%%%%%%%%%%%%%%%%%%%%%
\usepackage[T1]{fontenc}         % Use T1 encoding instead of OT1
\usepackage[utf8]{inputenc}      % Use UTF8 input encoding
\usepackage{microtype}           % Improve typography
\usepackage{booktabs}            % Publication quality tables
\usepackage{amsmath}
\usepackage{graphicx}
\usepackage{float}
\usepackage[exponent-product=\cdot]{siunitx}
\usepackage[colorlinks,breaklinks]{hyperref}
\hypersetup{linkcolor=black, citecolor=black, urlcolor=black}

\usepackage{lipsum}

\def\equationautorefname{Eq.}
\def\figureautorefname{Fig.}

%%%%%%%%%% MC JOURNAL NOTES %%%%%%%%%%%%%%%%%%%%%%%%%%%%%%%%%%%%%%%%%
%
% References can be typeset properly using the provided \textsc{Bib}\TeX style file. See examples of a journal~\cite{journal}, conference proceedings~\cite{proceedings}, book~\cite{book}, and miscellaneous~\cite{misc}.
%
%References to websites are discouraged, but acceptable if absolutely necessary.  It is the author?s responsibility to check links in the pdf file.
%Final PDF file size should be no more than 4 MB.  Recommended paper length is 10-12 pages.
%
% Subsection Title: First Character of Each Non-trivial Word is Uppercase
%Equations (Equation \ref{eqn:sample}) should be centered and sequentially numbered to the flush right of the formula.
% Sub-subsection level and lower: only first character uppercase
%
%\begin{equation}
%  1+1=2 \label{eqn:sample}
%\end{equation}
% The continuation of a paragraph after an equation is not indented
%
%Figures and tables should appear as closely as possible to where they are first cited, e.g. Fig. \ref{fig:sample}, in the text.  Figures are numbered in Arabic numerals, with the caption centered below the figure, in boldface. 
%
%\begin{figure}[H]
%  \centering
%  \includegraphics[width=3in]{figure.png}
%  \caption{Sample Figure. Color is permitted, but must be readable if printed.}
%  \label{fig:sample}
%\end{figure}
%
% When importing figures or any graphical image please verify two things:
% 1. Any number, text or symbol is no smaller than 10-point after reduction to the actual window in your paper;
% 2. That it can be translated into PDF.
%
% Tables, like Table \ref{tab:sample}, are numbered in Roman numerals, with the caption centered above the table, in \textbf{boldface}.  
% Double-space before and after the table.
%
%\begin{table}
%  \centering
%  \caption{Sample table: accuracy of nodal and characteristic methods}
%  \begin{tabular}{lcccc}
%    \toprule
%    Mesh & 8 x 8 & 16 x 16 & 32 x 32 & 64 x 64 \\
%    \midrule
%    Nodal & \num{1.000e-1} & \num{2.500e-2} & \num{6.250e-3} & \num%{1.563e-3} \\
%    Characteristic & \num{1.000e-1} & \num{2.500e-2} & \num{6.250e-3} & \num{1.563e-3} \\
%    \bottomrule
%  \end{tabular}
%  \label{tab:sample}
%\end{table}


%%%%%%%%%%%%%%%%%%%%%%%%%%%%%%%%%%%%%%%%%%%%%%%%%%%%%%%%%%%%%%%%%%%%%
% Insert authors' names and short version of title in lines below

\authorHead{Shaun Marshall, Blake Currier, Andrew D Hodgdon}
\shortTitle{Faraday Cup Proton Gain Simulation}


%%%%%%%%%%%%%%%%%%%%%%%%%%%%%%%%%%%%%%%%%%%%%%%%%%%%%%%%%%%%%%%%%%%%%
\begin{document}

\title{Proton Induced Gain in a Portable Faraday Cup}

\author{Shaun Marshall}
\author{Blake Currier}
\affil{
Department of Physics \\
Worcester Polytechnic Institute \\
100 Institute Rd, Worcester, MA 01609
}

\author{Andrew D Hodgdon, CHP}
\affil{
Radsim, LLC \\
584 Grove St, Newton, MA 02462 \\
adhodgdon@radsim.org
}

\maketitle

\begin{abstract}
A Faraday Cup (FC) is being designed to calibrate therapy-range proton accelerators, i.e., 50 to 250 MeV. The FC must be accurate to 1\% as well as portable, hence vacuum-less and low mass. The FC is a copper cylinder coated with kapton insulation and silver ground. The Monte Carlo method (MCNP6 and Geant4) was used to simulate the radiation cascade and predict gain versus height (H), diameter (D) and insulator thickness (K). H and D were mostly functions of proton range. Increasing either increases mass, reducing either increases proton leakage, hence decreases accuracy. Kapton functions to capture backscattered electrons, the function of the fields in a standard FC. Greater K increases capture but increases secondary electron in-leakage. Determining optimal K was made difficult by the lack of low energy proton, electron cross-sections. A secondary electron model was programmed with the SDEF command for the MCNP model based on recently published cross-section approximations. This secondary electron source method was benchmarked against a series of experimental measurements (by others) of protons on copper and on water. Three FCs were built, each with different values of K. They are currently being tested. 

\emph{Key Words}: Monte Carlo, Geant, MCNP, Faraday Cup
\end{abstract}


%%%%%%%%%%%%%%%%%%%%%%%%%%%%%%%%%%%%%%%%%%%%%%%%%%%%%%%%%%%%%%%%%%%%%
\section{Introduction}

In modern day radiation therapy, protons have become one increasingly popular method of treating cancer near critical structures, with many dosimetric advantages of charged particle interactions (References). A novel portable vacuumless Faraday Cup measuring device was designed to calibrate proton therapy facilities, in energies ranging from 50 to 250 MeV. The detector is constructed of a copper cylinder, coated with kapton insulation and grounded with silver (Reference). Monte Carlo computational simulations in MCNP6 (Reference) and GEANT4 (Referenced) were performed to evaluate radiation cascade effects and predict signal versus height, diameter and insulator thickness characteristics.

Preliminary results indicated that increasing the mass of the Faraday Cup’s conductor reduced proton leakage but increased the system accuracy. While a greater kapton thickness increases the capture of primary and secondary electrons, it also increases secondary electron leakage. Additionally, determining the optimal kapton thickness has been made difficult by the lack of low energy proton and electron cross-sections in current Monte Carlo based simulation programs (Reference). A comprehensive secondary electron evaluation of the kapton was performed and benchmarked against a series of experimental measurements by Borovsky et al.1 and J. Gordon et al (Reference).

In conjunction with the computational calculations, three (3) prototype Faraday Cup measuring devices were constructed by Pyramid Technical Consultants, Inc. (Waltham, Ma), each having a different thickness of kapton. The units were tested in Germany to determine accuracy of the new design. 


%%%%%%%%%%%%%%%%%%%%%%%%%%%%%%%%%%%%%%%%%%%%%%%%%%%%%%%%%%%%%%%%%%%%%
\section{Experimental Background}


\subsection{Heidelberg Institute of Technology}

Table \ref{tab:HIT_data}

\begin{table}
  \centering
  \caption{Measured Gain from HIT Beam Stops}
  \begin{tabular}{lccccc}
    \toprule
    Energy (MeV) & S59 & S100 & S200 \\
    \midrule
    70.03  & \num{0.9750} & \num{0.9385} & \num{0.9350} \\
    100.46 & \num{0.9850} & \num{0.9500} & \num{0.9475} \\
    130.52 & \num{0.9925} & \num{0.9580} & \num{0.9525} \\
    160.09 & \num{1.0000} & \num{0.9635} & \num{0.9590} \\
    190.48 & \num{1.0075} & \num{0.9715} & \num{0.9650} \\
    221.06 & \num{1.0125} & \num{0.9800} & \num{0.9770} \\
    \bottomrule
  \end{tabular}
  \label{tab:HIT_data}
\end{table}

\subsection{GO14}


\subsection{BO88}


\section{Simulation Results}


\subsection{MCNP6}
(LATER: from ADH)

\subsection{Geant4}

Geant4 is an object-oriented C++ toolkit for developing applications which simulate the passage of particles through matter. Libraries of cross-section tables, elemental/molecular properties, and pre-defined stochastic physics processes allow for rapid, intuitive invocation of necessary system setup commands. Once initialized, \lq\lq Manager" modules cooperate to organize and accumulate dynamic information which is organized in the following chronology:

\begin{enumerate}
\item The \textbf{DetectorConstruction} class is called to verify, store and lock the predefined geometry.
\item The \textbf{G4UIManager} initializes upon successful compilation and execution of the \emph{main()} routine.  If a visualizer is selected, \textbf{G4VisManager} is also invoked.
\item The user issues the command to execute a macro file of \emph{runs}; each run is characterized by the defined beam particle type, the beam energy, and the number of \emph{events}, or number of such isolated simulations.  If multithreading is available, \textbf{G4RunManager} allocates the events to the available worker threads on a rolling basis.
\item For each event, the simulation of the \emph{primary} (beam) particle proceeds, constructing a new \emph{track}, or well-defined trajectory for every particle not at rest.
\item The behavior of every track is determined dynamically, with each \emph{step}, or stochastically occuring physical process (collisions, absorbtions, etc) of the particle in some medium.
\end{enumerate}

\subsubsection{UserAction methods}

A useful feature of Geant4 is the ability to create user-defined actions (methods) throughout each module, which allows for a very fine-tuned analysis throughout the entire simulation.  The following summarizes the custom details and methods for this application

\begin{itemize}
\item \textbf{DetectorConstruction.cc:} A copper cylinder of radius 3 cm and height 10 cm is covered in Kapton film of varying thickenesses: 59 $\mu$m, 100 $\mu$m and 200 $\mu$m.  The film thickness is iterated by a function which is called before the command macro is examined.  The top face of the copper lies in the $z=0$ plane, with the beam approach the system from beneath.
\item \textbf{SteppingAction.cc:} [For every step,] immediately checks if the step is the finale of a track.  If so, the particle's vertex (original position) and destination volume and coordinates are found, and a charge signal calculation occurs.  Entering/Leaving the copper gives a net signal of $\pm q$ where $q$ is the charge of the particle.  Entering/Leaving Kapton gives a relative proportionality of

\begin{equation}
s_{q\rightleftharpoons KA} = \pm q\times\max\left[r_{\%}, z_{\%}\right], \label{eqn:s_KA}
\end{equation}

where $r_{\%}$ is the percent distance away from the copper radially and $z_{\%}$ is the percent distance away laterally.  The signals are grouped and saved by a unique eventID number.
\item \textbf{EventAction.cc:} At the end of each event, the signals are tallied, grouped, and saved by a unique runID number.
\item \textbf{RunAction.cc:} At the end of each run, the average and standard deviation of the signals are acquired.
\end{itemize}

\subsubsection{Experimental parameters}

Table \ref{tab:geant4setup} summarizes the detector geometry of each run.  The order of logical volume layers starting from the innermost are 1) the copper cylinder, 2) the Kapton1 film, 3) the silver paint layer, and 4) the Kapton2 film.  Constructing cylindrical \lq\lq layers" is as straightforward as defining a cylinder within another's logical volume.  Data were acquired as a function of impinging proton energy using the 50-250 MeV range as used in the HIT experiment.  The Kapton1 thickness optimization was applied to this model both with and without the silver and secondary Kapton (\emph{+Ag/KA}).

\begin{table}
  \centering
  \caption{Geant4 Simulation Cylindrical Construction}
  \begin{tabular}{ccc}
    \toprule
    Volume  & Radius (mm) & Height (mm) \\
    \midrule
    Copper  & \num{30} & \num{100} \\
    \toprule\toprule
            & Model    & Thickness ($\mu$m) \\
    \midrule
    Kapton1 & S59      & \num{59}  \\
            & S100     & \num{100} \\
            & S200     & \num{200} \\
    Silver  & +Ag/KA   & \num{12}  \\
    Kapton2 & +Ag/KA   & \num{62}  \\
    \bottomrule
  \end{tabular}
  \label{tab:geant4setup}
\end{table}


%%%%%%%%%%%%%%%%%%%%%%%%%%%%%%%%%%%%%%%%%%%%%%%%%%%%%%%%%%%%%%%%%%%%%
\section{Results}

The units of signal gain are $\frac{Q_net}{Q_P}$, where $Q_{net}$ is the net transfer of charge into the cup, and $Q_P$ is the net charge of the million protons irradiating the cup.  Charges entering and leaving the primary Kapton film covering the copper are subject to the linear proportion behavior defined in Eq.~\ref{eqn:s_KA}.  Table \ref{tab:geant4_data} shows a sample output of each model in both air and vaccuum, the latter to remove oversaturation of beta emissions from the air due to delta-ray production (LATER: citation needed).

\begin{table}
  \centering
  \caption{Predicted Gain from High-Energy Protons using Geant4}
  \begin{tabular}{lccccc}
    \toprule
    Model & Energy (MeV) & (-Ag/KA) & (-Ag/KA) \emph{in vacuo} & (+Ag/KA) & (+Ag/KA) \emph{in vacuo} \\
    \midrule
    S59 & 70.03  & \num{0.953588} & \num{1.00036} & \num{0.983320} & \emph{in progress...} \\
        & 100.46 & \num{0.967417} & \num{1.00068} & \num{0.982263} \\
        & 130.52 & \num{0.975593} & \num{1.00118} & \num{0.982531} \\
        & 160.09 & \num{0.981094} & \num{1.00177} & \num{0.983127} \\
        & 190.48 & \num{0.985111} & \num{1.00238} & \num{0.984641} \\
        & 221.06 & \num{0.988151} & \num{1.00314} & \num{0.986131} \\
        & 250.00 & \num{0.990298} & \num{1.00358} & \num{0.987536} \\
    \midrule
    S100 & 70.03 & \num{0.953827} & \num{1.00036} & \num{0.983731} \\
        & 100.46 & \num{0.966795} & \num{1.00072} & \num{0.982408} \\
        & 130.52 & \num{0.975725} & \num{1.00121} & \num{0.982508} \\
        & 160.09 & \num{0.981055} & \num{1.00180} & \num{0.983059} \\
        & 190.48 & \num{0.985189} & \num{1.00245} & \num{0.984910} \\
        & 221.06 & \num{0.988149} & \num{1.00326} & \num{0.986215} \\
        & 250.00 & \num{0.990324} & \num{1.00349} & \num{0.987278} \\
    \midrule
    S200 & 70.03 & \num{0.954372} & \num{1.00037} & \num{0.983544} \\
        & 100.46 & \num{0.966915} & \num{1.00068} & \num{0.982554} \\
        & 130.52 & \num{0.975377} & \num{1.00126} & \num{0.982246} \\
        & 160.09 & \num{0.980998} & \num{1.00178} & \num{0.983405} \\
        & 190.48 & \num{0.985217} & \num{1.00244} & \num{0.984706} \\
        & 221.06 & \num{0.988312} & \num{1.00320} & \num{0.986402} \\
        & 250.00 & \num{0.990213} & \num{1.00343} & \num{0.987178} \\
    \bottomrule
  \end{tabular}
  \label{tab:geant4_data}
\end{table}


%%%%%%%%%%%%%%%%%%%%%%%%%%%%%%%%%%%%%%%%%%%%%%%%%%%%%%%%%%%%%%%%%%%%%
\section{Discussion}


%%%%%%%%%%%%%%%%%%%%%%%%%%%%%%%%%%%%%%%%%%%%%%%%%%%%%%%%%%%%%%%%%%%%%
\section{Conclusions}


%%%%%%%%%%%%%%%%%%%%%%%%%%%%%%%%%%%%%%%%%%%%%%%%%%%%%%%%%%%%%%%%%%%%%
\section{Acknowledgments}

We would like to express our sincerest gratitude to Paul Romano and Tom Sutton, who provided the template for this paper.

%%%%%%%%%%%%%%%%%%%%%%%%%%%%%%%%%%%%%%%%%%%%%%%%%%%%%%%%%%%%%%%%%%%%%
\setlength{\baselineskip}{12pt}

\bibliographystyle{mc2015}
\bibliography{references}

%%%%%%%%%%%%%%%%%%%%%%%%%%%%%%%%%%%%%%%%%%%%%%%%%%%%%%%%%%%%%%%%%%%%%
\appendix
\section{}

Code bits?

\pageref{lastpage}
\end{document}
