\documentclass{mc2015}

%%%%%%%%%%%%%%%%%%%%%%%%%%%%%%%%%%%%%%%%%%%%%%%%%%%%%%%%%%%%%%%%%%%%%
\usepackage[T1]{fontenc}         % Use T1 encoding instead of OT1
\usepackage[utf8]{inputenc}      % Use UTF8 input encoding
\usepackage{microtype}           % Improve typography
\usepackage{booktabs}            % Publication quality tables
\usepackage{amsmath}
\usepackage{graphicx}
\usepackage{caption}
\usepackage{subcaption}
\usepackage{float}
\usepackage[exponent-product=\cdot]{siunitx}
\usepackage[colorlinks,breaklinks]{hyperref}
\hypersetup{linkcolor=black, citecolor=black, urlcolor=black}

\def\equationautorefname{Eq.}
\def\figureautorefname{Fig.}

%%%%%%%%%% MC JOURNAL NOTES %%%%%%%%%%%%%%%%%%%%%%%%%%%%%%%%%%%%%%%%%
%
% References can be typeset properly using the provided \textsc{Bib}\TeX style file. See examples of a journal~\cite{journal}, conference proceedings~\cite{proceedings}, book~\cite{book}, and miscellaneous~\cite{misc}.
%
%References to websites are discouraged, but acceptable if absolutely necessary.  It is the author?s responsibility to check links in the pdf file.
%Final PDF file size should be no more than 4 MB.  Recommended paper length is 10-12 pages.
%
% Subsection Title: First Character of Each Non-trivial Word is Uppercase
%Equations (Equation \ref{eqn:sample}) should be centered and sequentially numbered to the flush right of the formula.
% Sub-subsection level and lower: only first character uppercase
%
%\begin{equation}
%  1+1=2 \label{eqn:sample}
%\end{equation}
% The continuation of a paragraph after an equation is not indented
%
%Figures and tables should appear as closely as possible to where they are first cited, e.g. Fig. \ref{fig:sample}, in the text.  Figures are numbered in Arabic numerals, with the caption centered below the figure, in boldface. 
%
%\begin{figure}[H]
%  \centering
%  \includegraphics[width=3in]{figure.png}
%  \caption{Sample Figure. Color is permitted, but must be readable if printed.}
%  \label{fig:sample}
%\end{figure}
%
% When importing figures or any graphical image please verify two things:
% 1. Any number, text or symbol is no smaller than 10-point after reduction to the actual window in your paper;
% 2. That it can be translated into PDF.
%
% Tables, like Table \ref{tab:sample}, are numbered in Roman numerals, with the caption centered above the table, in \textbf{boldface}.  
% Double-space before and after the table.
%
%\begin{table}
%  \centering
%  \caption{Sample table: accuracy of nodal and characteristic methods}
%  \begin{tabular}{lcccc}
%    \toprule
%    Mesh & 8 x 8 & 16 x 16 & 32 x 32 & 64 x 64 \\
%    \midrule
%    Nodal & \num{1.000e-1} & \num{2.500e-2} & \num{6.250e-3} & \num%{1.563e-3} \\
%    Characteristic & \num{1.000e-1} & \num{2.500e-2} & \num{6.250e-3} & \num{1.563e-3} \\
%    \bottomrule
%  \end{tabular}
%  \label{tab:sample}
%\end{table}


%%%%%%%%%%%%%%%%%%%%%%%%%%%%%%%%%%%%%%%%%%%%%%%%%%%%%%%%%%%%%%%%%%%%%
% Insert authors' names and short version of title in lines below

\authorHead{Shaun Marshall, Blake Currier, Andrew D Hodgdon}
\shortTitle{Proton Gain in Faraday Cup}


%%%%%%%%%%%%%%%%%%%%%%%%%%%%%%%%%%%%%%%%%%%%%%%%%%%%%%%%%%%%%%%%%%%%%
\begin{document}

\title{Proton-Induced Gain in a Portable Faraday Cup}

\author{Shaun Marshall}
\author{Blake Currier}
\affil{
Department of Physics \\
Worcester Polytechnic Institute \\
100 Institute Rd, Worcester, MA 01609
}

\author{Andrew D Hodgdon, CHP}
\affil{
		Radsim, LLC \\
584 Grove St, Newton, MA 02462 \\
adhodgdon@radsim.org
}

\maketitle

\begin{abstract}
A Portable Faraday Cup (PFC) is being designed to calibrate therapy-range proton accelerators (50 to 250 MeV).  The PFC must be accurate to 1\% and practical, hence vacuum-less and of low mass.  Copper was chosen as the detector core, coated with a Kapton insulating film and silver ground.  The Monte Carlo method (MCNP6 and Geant4) was used to simulate the radiation cascade and predict gain versus height (H), diameter (D) and insulator thickness (K). H and D were mostly functions of proton range; both are proportional to mass and inversely so to proton leakage, and thus decreases detector accuracy. Kapton functions to capture backscattered electrons, the function of the fields in a standard Faraday Cup. Greater K increases capture but increases secondary electron in-leakage. Determining optimal K was made difficult by the lack of low energy proton and electron cross-sections. A secondary electron model was programmed with the SDEF command for the MCNP model based on recently published cross-section approximations. This secondary electron source method was benchmarked against a series of experimental measurements of protons on copper and on water.

\emph{Key Words}: Monte Carlo, Geant4, MCNP6, Faraday Cup, Proton Beam, Secondary Electron Emission
\end{abstract}


%%%%%%%%%%%%%%%%%%%%%%%%%%%%%%%%%%%%%%%%%%%%%%%%%%%%%%%%%%%%%%%%%%%%%
\section{Introduction}

In modern radiation therapy, protons have become an increasingly popular method of treating cancer near critical structures, with many dosimetric advantages of charged particle interactions\cite{ne05,ryckman}. A novel, portable, vacuumless Faraday Cup for detecting charged particles was designed to calibrate proton therapy facilities, in energies ranging from 50 to 250 MeV. The detector is constructed of a copper cylinder, coated with Kapton insulation and grounded with silver (CITE). Monte Carlo computational simulations in MCNP6 (CITE) and GEANT4 (CITE) were performed to evaluate radiation cascade effects and predict signal versus height, diameter and insulator thickness characteristics.

Preliminary results indicated that increasing the mass of the Faraday Cup’s conductor reduced proton leakage but increased the system accuracy. While additional Kapton captures more primary and secondary electrons, it also increases secondary electron leakage into the copper. Optimizing this Kapton thickness has been made difficult by the lack of low energy proton and electron cross-sections in current Monte Carlo based simulation programs (CITE). A comprehensive secondary electron evaluation of the Kapton was performed and benchmarked against a series of experimental measurements by J. Gordon et al \cite{PTC-HIT}.

In corroboration with the computational calculations, three prototype Faraday Cup devices were constructed by Pyramid Technical Consultants, Inc. (Waltham, Ma), each with a different thickness of Kapton. The units were tested in Germany to determine accuracy of the new design.


%%%%%%%%%%%%%%%%%%%%%%%%%%%%%%%%%%%%%%%%%%%%%%%%%%%%%%%%%%%%%%%%%%%%%
\section{Experimental Background}

Limited empirical evidence of proton-beam dose was available before the gold/aluminum-oxide Faraday Cup emulations at the Los Alamos National Laboratory, which produced charged-particle beam current yields as a function of (at the time, the highest) beam energies from 5-24 MeV; it was found that secondary electron yield varied inversely with impinging proton beam energy, fitting an inverse-root curve (as predicted by Sternglass) with approximately 10\% error.\cite{bo88,st57}.  This was later semi-empirically validated with similar models\cite{Castaneda} and simulation using a constant-proportionality estimate of the secondary electron yield with the stopping power of protons in respective materials\cite{Kramer}; however, evidence exists which questions the validity of the latter approximation\cite{du02}.  Though simulation techniques struggle to converge to comparable values for low beam energy scenarios, the findings establish a reasonable basis in setting the Faraday Cup depth to extend further than the material-specific range of the the particle.  This removes the possibility of transmission-sputtering and second-surface electron emission occurences, providing the user with a more controlled ammeter measurement\cite{bo88}.

A unique approach to the Faraday Cup, a series of copper sheets separated by Kapton insulators, was tested in the Harvard Cyclotron Laboratory at a much higher energy.  This \lq\lq Multilayer" prototype provided empirical data to benchmark the applicability of separate hadronic interaction modules of Geant3.2.1. Each sheet of Kapton offers a mid-range, low-interaction field within which secondary electrons (or other stray charged particles) and their abandoned ion may remain bound to each other\cite{go99}.  This novel approach of \lq\lq trapping" the electrons which would not otherwise be impeded by the dense metal localizes the effects of Coulombic scattering and normalizes the measured total signal of the charged particles scattered per unit length and beam input; a gain differential of depth, independent of contributions of secondary electrons in the air.  The role of the vacuum to constrain particle flux was transferred to a silver external grounding brace in the experiments carried out by the Heidelberg Institute of Technology in a recent development to incorporate portability in measuring device.  Current efforts are directed at establishing proof-of-concept theoretically with simulated reproduction.

%%%%%%%%%%%%%%%%%%%%%%%%%%%%%%%%%%%%%%%%%%%%%%%%%%%%%%%%%%%%%%%%%%%%%
\section{Simulation Methodology}

A variety of adaptations to the model were examined with both MCNP6 and Geant4 Monte Carlo software, the latter across a high-energy spectrum 

\subsection{MCNP6}

MCNP version 6.1 with standard cross-section libraries was used to simulate gain in a solid copper cylinder with varying diameter and Kapton film thickness, and grounded with a layer of silver; the geometry is shown in Fig.~\ref{fig:model_geometry}. The height of the cylinder is fixed at 10 cm, the diameter is varied from 2 to 10 cm and the Kapton thickness is varied from 25 to 75 microns. The materials are standard copper, Kapton, silver and air at STP. The source is a 2 cm diameter proton beam at 250 MeV. This is the maximum expected energy. A suitable diameter for this energy will be suitable for lower energies. Physics was turned on for seven particles: neutrons, photons, electrons, protons, deuterons, tritons and alphas.

\begin{figure}[H]
  \centering
  \includegraphics[width=5in]{figures/fig_geometry.png}
  \caption{Geometry from side of horizontal copper cylinder}
  \label{fig:model_geometry}
\end{figure}

Gain was modeled simplistically by tallies of two charged particles (protons and electrons) crossing three surfaces (face, side and end). Gain is expressed in terms of elemental charge deposited in copper per beam proton, shown in eq. \ref{eqn:gain_definition}.

\begin{equation}
Gain = \frac{\sum Q_{particle\rightarrow Cu} - \sum Q_{particle\leftarrow Cu}}{\sum Q_{P+\ beam}}
\label{eqn:gain_definition}
\end{equation}

MCNP6 distinguishes between two types of secondary electrons, those that are created by proton collisions, i.e., SE$_h$, (MCNP cannot track such electrons), and those that are generated by other particles (notably photons that are themselves secondary), i.e., SE$_{!h}$. The following assumptions were made in this model:

\begin{itemize}
\item For the selection of a good copper diameter the SE$_h$ are not important (untested)
\item Protons and electrons are the only contributors to gain (valid within range of 2E-4)
\item The choice of copper diameter has little effect on the detailed behavior of SE$_{!h}$ electrons in Kapton (i.e., capture causing mirror charge in copper). This means that the location and energy of electrons captured in the kapton do not have to be tallied to get valid answer for the copper diameter (untested)
\end{itemize}

The detector is to be a beam proton counter. The perfect detector will yield a gain of unity, seen in eq. \ref{eqn:error_definition}.

\begin{equation}
Error = (Gain-1)\cdot100\%
\label{eqn:error_definition}
\end{equation}

\subsection{Geant4}

Geant4 is an object-oriented C++ toolkit for developing applications which simulate the passage of particles through matter. Libraries of cross-section tables, elemental/molecular properties, and pre-defined stochastic physics processes allow for rapid, intuitive invocation of necessary system setup commands. Once initialized, \lq\lq Manager" modules cooperate to organize and accumulate dynamic information of all affected particles.  Each \emph{event} (beam iteration) in a \emph{run} (predefined geometry and physics for a series of events) stores the resulting \emph{tracks} of particles which gained momentum, each for some number of \emph{steps} where the defined physics governs the future pathway.

\subsubsection{UserAction methods}

A useful feature of Geant4 is the ability to create user-defined actions (methods) throughout each module, which allows for a very fine-tuned analysis throughout the entire simulation.  The following summarizes the custom details and methods for this application

\begin{itemize}
\item \textbf{DetectorConstruction.cc:} A copper cylinder of radius 3 cm and height 10 cm is covered in Kapton film of varying thickenesses: 59 $\mu$m, 100 $\mu$m and 200 $\mu$m.  The film thickness is iterated by a function which is called before the command macro is examined.  The top face of the copper lies in the $z=0$ plane, with the beam approach the system from beneath.
\item \textbf{SteppingAction.cc:} [For every step,] immediately checks if the step is the finale of a track.  If so, the particle's vertex (original position) and destination volume and coordinates are found, and a charge signal calculation occurs.  Entering/Leaving the copper gives a net signal of $\pm q$ where $q$ is the charge of the particle.  Entering/Leaving Kapton gives a relative proportionality of

\begin{equation}
s_{q\rightleftharpoons KA} = \pm q\times\max\left[r_{\%}, z_{\%}\right], \label{eqn:s_KA}
\end{equation}

where $r_{\%}$ is the percent distance away from the copper radially and $z_{\%}$ is the percent distance away laterally.  The signals are grouped and saved by a unique eventID number.
\item \textbf{EventAction.cc:} At the end of each event, the signals are tallied, grouped, and saved by a unique runID number.
\item \textbf{RunAction.cc:} At the end of each run, the average and standard deviation of the signals are acquired.
\end{itemize}

\subsubsection{Experimental parameters}

Table \ref{tab:geant4setup} summarizes the detector geometry of each run.  The order of logical volume layers starting from the innermost are 1) the copper cylinder, 2) the Kapton1 film, 3) the silver paint layer, and 4) the Kapton2 film.  Constructing cylindrical \lq\lq layers" is as straightforward as defining a cylinder within another's logical volume.  Data were acquired as a function of impinging proton energy using the 50-250 MeV range as used in the HIT experiment.  The Kapton1 thickness optimization was applied to this model both with and without the silver and secondary Kapton (\emph{+Ag/KA}).

\begin{table}[H]
  \centering
  \caption{Geant4 Simulation Cylindrical Construction}
  \begin{tabular}{ccc}
    \toprule
    Volume  & Radius (mm) & Height (mm) \\
    \midrule
    Copper  & \num{30} & \num{100} \\
    \toprule\toprule
            & Model    & Thickness ($\mu$m) \\
    \midrule
    Kapton1 & S59      & \num{59}  \\
            & S100     & \num{100} \\
            & S200     & \num{200} \\
    Silver  & +Ag/KA   & \num{12}  \\
    Kapton2 & +Ag/KA   & \num{62}  \\
    \bottomrule
  \end{tabular}
  \label{tab:geant4setup}
\end{table}


%%%%%%%%%%%%%%%%%%%%%%%%%%%%%%%%%%%%%%%%%%%%%%%%%%%%%%%%%%%%%%%%%%%%%
\section{Results}



\subsection{MCNP6 Simulation}

\begin{figure}[H]
  \centering
  \includegraphics[width=4in]{figures/fig_error_diameter.png}
  \caption{Signal Error vs Diameter and Simulation Method}
  \label{fig:error_diameter}
\end{figure}

Fig.~\ref{fig:error_diameter} shows the variation of error with copper diameter and method. Two methods are compared, SRIM (reference) and MCNP6. A conservative estimate of the error is found to drop below 1\% beyond a 6 cm diameter.  The same calculation was repeated with proton tallies alone, i.e., without any secondary electrons. This shows that if electrons had been completely ignored, the MCNP6 results would have been similar to the SRIM model and a diameter of 8 cm would have seemed reasonable.

The distribution of proton collisions from 50 beam protons is in Fig.~\ref{fig:mcnp_tracks}. The distribution of 6 other particles (born of 50 Protons) is in Fig.~\ref{fig:mcnp_dist}; neutrons go everywhere. The distribution of electrons (just type SE$_{!h}$) is similar to photons. Table \ref{tab:mcnp_neutral_crossing} shows the boundary crossings of neutral particles. In the problem there are about two neutral particle boundary crossings per beam proton. Table \ref{tab:mcnp_charge_crossing} shows the breakdown of gain from various charged particles for given directions and copper surfaces. This assumes 250 MeV Proton Beam, 6cm Copper Diameter and 50 microns kapton. The effect of secondary electrons produced directly by proton collisions (SD$_h$) is not included.

\begin{figure}[H]
  \centering
  \includegraphics[width=4in]{figures/fig_mcnp_tracks.png}
  \caption{Distribution of 50 Protons of Energy 250 MeV using MCNP6. Note the singular backscatter.}
  \label{fig:mcnp_tracks}
\end{figure}

\begin{figure}[H]
        \centering
        \begin{subfigure}[b]{0.2\textwidth}
                \includegraphics[width=\textwidth]{figures/fig_mcnp_dist_n.png}
                \label{fig:mcnp_dist_n}
        \end{subfigure}
        %
        \begin{subfigure}[b]{0.2\textwidth}
                \includegraphics[width=\textwidth]{figures/fig_mcnp_dist_ph.png}
                \label{fig:mcnp_dist_ph}
        \end{subfigure}
        %
        \begin{subfigure}[b]{0.2\textwidth}
                \includegraphics[width=\textwidth]{figures/fig_mcnp_dist_e.png}
                \label{fig:mcnp_dist_e}
        \end{subfigure}
        
        \begin{subfigure}[b]{0.2\textwidth}
                \includegraphics[width=\textwidth]{figures/fig_mcnp_dist_d.png}
                \label{fig:mcnp_dist_d}
        \end{subfigure}
        %
        \begin{subfigure}[b]{0.2\textwidth}
                \includegraphics[width=\textwidth]{figures/fig_mcnp_dist_t.png}
                \label{fig:mcnp_dist_t}
        \end{subfigure}
        %
        \begin{subfigure}[b]{0.2\textwidth}
                \includegraphics[width=\textwidth]{figures/fig_mcnp_dist_a.png}
                \label{fig:mcnp_dist_a}
        \end{subfigure}
        \caption{Distribution of six other particles}
        \label{fig:mcnp_dist}
\end{figure}

It is worth examining the behavior of the model. Kapton thickness (25, 50 and 75 microns) seems to make no difference in gain. Neither do the inclusion of tallies for deuterons, tritons and alphas, although the MCNP output file notes the absence of production cross-sections for these particles\footnote{The absence of ion production XS noted in the MCNP output file vs the presence of ions in the run needs to be checked out.}. All of the figures and tables below were for the case of 6cm diameter, 50 microns of kapton and 50 beam protons of energy 250 MeV.

\begin{table}[H]
  \centering
  \caption{Charges Crossing Copper Surfaces (fraction per proton source) assuming 250 MeV Proton Beam, 6 cm Copper diameter, 50 microns of Kapton, and no SE$_\text{h}$}
  \begin{tabular}{clcccc}
    \toprule
    Particle & tally & face & cylinder & end & total \\
    \midrule
    n  & in & \num{0.00065} & \num{0.00044} & \num{0.10692} & \num{0.10801} \\
       & out & \num{0.14874} & \num{0.67867} & \num{0.00007} & \num{0.82747} \\
    \midrule
    $\gamma$ & in & \num{0.00038} & \num{0.00070} & \num{0.00011} & \num{0.00119} \\
             & out & \num{0.17603} & \num{0.65051} & \num{0.07812} & \num{0.90466} \\
    \bottomrule
  \end{tabular}
  \label{tab:mcnp_neutral_crossing}
\end{table}

\begin{table}[H]
  \centering
  \caption{Charges Crossing Copper Surfaces (fraction per proton source) assuming 250 MeV Proton Beam, 6 cm Copper diameter, 50 microns of Kapton, and no SE$_\text{h}$}
  \begin{tabular}{clcccc}
    \toprule
    Particle & tally & face & cylinder & end & total \\
    \midrule
       & in & \num{0.99997} & \num{0.00000} & \num{0.00000} & \num{0.99997} \\
    P+ & out & \num{0.00055} & \num{0.00123} & \num{0.00023} & \num{0.00201} \\
       & total & \num{0.99941} & \num{0.00123} & \num{0.00023} & \num{0.99796} \\
    \midrule
       & in & \num{0.00031} & \num{0.00087} & \num{0.00010} & \num{0.00128} \\
    E\ (no SE$_\text{h}$) & out & \num{0.00149} & \num{0.00450} & \num{0.00051} & \num{0.00650} \\
       & total & \num{0.00119} & \num{0.00362} & \num{0.00042} & \num{0.00523} \\
    \midrule
       & in & \num{0.00003} & \num{0.00000} & \num{0.00000} & \num{0.00003} \\
     d & out & \num{0.00007} & \num{0.00006} & \num{0.00002} & \num{0.00015} \\
       & total & \num{0.00004} & \num{0.00006} & \num{0.00002} & \num{0.00012} \\
    \midrule
       & in & \num{0.00002} & \num{0.00000} & \num{0.00000} & \num{0.00002} \\
     t & out & \num{0.00003} & \num{0.00006} & \num{0.00002} & \num{0.00003} \\
       & total & \num{0.00001} & \num{0.00006} & \num{0.00002} & \num{0.00001} \\
    \midrule
       & in & \num{0.00003} & \num{0.00000} & \num{0.00000} & \num{0.00003} \\
     a & out & \num{0.00003} & \num{0.00000} & \num{0.00000} & \num{0.00003} \\
       & total & \num{0.00000} & \num{0.00000} & \num{0.00000} & \num{0.00000} \\
    \midrule
       & in & \num{0.99974} & \num{0.00087} & \num{0.00010} & \num{1.00124} \\
    Signal & out & \num{0.00082} & \num{0.00320} & \num{0.00026} & \num{0.00851} \\
       & total & \num{0.99823} & \num{0.00485} & \num{0.00065} & \num{0.99273} \\
    \bottomrule
  \end{tabular}
  \label{tab:mcnp_charge_crossing}
\end{table}

\subsection{Geant4 Simulation}

Signal gain is defined in eq.~\ref{eqn:gain_definition}.  Charges entering and leaving the primary Kapton film covering the copper are subject to the linear proportion behavior defined in Eq.~\ref{eqn:s_KA}.  Table \ref{tab:geant4_data} shows a sample output of each model in both air and vaccuum, the latter to remove oversaturation of beta emissions from the air due to delta-ray production (LATER: citation needed).

\begin{table}[H]
  \centering
  \caption{Predicted Gain from High-Energy Protons using Geant4}
  \begin{tabular}{lccccc}
    \toprule
    Model & Energy (MeV) & (-Ag/KA) & (-Ag/KA) \emph{in vacuo} & (+Ag/KA) & (+Ag/KA) \emph{in vacuo} \\
    \midrule
    S59 & 70.03  & \num{0.953588} & \num{0.974846} & \num{0.963133} & \num{0.963262} \\
        & 100.46 & \num{0.967417} & \num{0.984694} & \num{0.970072} & \num{0.970867} \\
        & 130.52 & \num{0.975593} & \num{0.990117} & \num{0.974286} & \num{0.976064} \\
        & 160.09 & \num{0.981094} & \num{0.994044} & \num{0.978484} & \num{0.980236} \\
        & 190.48 & \num{0.985111} & \num{0.996718} & \num{0.981775} & \num{0.983519} \\
        & 221.06 & \num{0.988151} & \num{0.999012} & \num{0.984790} & \num{0.986344} \\
        & 250.00 & \num{0.990298} & \num{1.000260} & \num{0.986376} & \num{0.987953} \\
    \midrule
    S100 & 70.03 & \num{0.953827} & \num{0.974725} & \num{0.962994} & \num{0.963440} \\
        & 100.46 & \num{0.966795} & \num{0.984533} & \num{0.970114} & \num{0.970319} \\
        & 130.52 & \num{0.975725} & \num{0.990464} & \num{0.974399} & \num{0.980993} \\
        & 160.09 & \num{0.981055} & \num{0.994167} & \num{0.978401} & \num{0.976085} \\
        & 190.48 & \num{0.985189} & \num{0.996801} & \num{0.981909} & \num{0.980045} \\
        & 221.06 & \num{0.988149} & \num{0.999164} & \num{0.984451} & \num{0.983287} \\
        & 250.00 & \num{0.990324} & \num{1.000160} & \num{0.986188} & \num{0.988118} \\
    \midrule
    S200 & 70.03 & \num{0.954372} & \num{0.974735} & \num{0.962984} & \num{0.963351} \\
        & 100.46 & \num{0.966915} & \num{0.984373} & \num{0.970077} & \num{0.971036} \\
        & 130.52 & \num{0.975377} & \num{0.990337} & \num{0.974646} & \num{0.976159} \\
        & 160.09 & \num{0.980998} & \num{0.994016} & \num{0.978510} & \num{0.980092} \\
        & 190.48 & \num{0.985217} & \num{0.996776} & \num{0.981840} & \num{0.983659} \\
        & 221.06 & \num{0.988312} & \num{0.999045} & \num{0.984644} & \num{0.986081} \\
        & 250.00 & \num{0.990213} & \num{0.999940} & \num{0.986601} & \num{0.988010} \\
    \bottomrule
  \end{tabular}
  \label{tab:geant4_data}
\end{table}

Fig.~\ref{fig:G4_dist} depicts the tracks of particles given the simulation of 50 250 MeV protons entering the S59 model.  The track color corresponds to particle charge, red for negative, blue for positive, and green neutral.  As observed in the MCNP6 simulation, neutrons are scattered everywhere; for the most part, electrons created in the Faraday Cup do not travel far, as expected given their low-energy production and high stopping-power in copper. Fig.~\ref{fig:comp_results} compares the signal gain in each model in both air and a vacuum, along with a pure copper control.

\begin{figure}[H]
  \centering
  \includegraphics[width=5in]{figures/fig_G4_dist.png}
  \caption{Distribution of 50 Protons of Energy 250 MeV using Geant4.}
  \label{fig:G4_dist}
\end{figure}

\begin{figure}[h]
  \centering
  \includegraphics[width=5in]{figures/fig_G4_results.png}
  \caption{Comparison of Geant4 output and HIT measurement.  Generated with Gnuplot.}
  \label{fig:comp_results}
\end{figure}

In agreement with (LATER), by replacing the air with a vacuum in any model, we remove the negative ions entering the cylinder from delta ray production, resulting in a greater positive charge measured as gain.  The addition of the silver-Kapton layer to the copper-Kapton core does have an apparent negative contribution to the gain reading, and requires a greater beam energy to establish accuracy.  The insulating Kapton thickness gave uncorrelated changes to the signal reading, but as shown from the copper contol run, these changes are not negligible.

%%%%%%%%%%%%%%%%%%%%%%%%%%%%%%%%%%%%%%%%%%%%%%%%%%%%%%%%%%%%%%%%%%%%%
\section{Discussion}

It is known that MCNP does not track electrons secondary to protons. Therefore, there are delta rays and secondary electrons from protons that have not been accounted for. This means that the error is greater than has been portrayed and needs further investigation because it is shown in Fig.~\ref{fig:error_diameter} that electron production (just the SE$_{!h}$) is impacted by the diameter on the choice of diameter.

The electrons that are included are those that are secondary to the particle cascade subsequent to protons. For example, there are electrons secondary to photons that come from neutrons that come from protons. Table \ref{tab:mcnp_neutral_crossing} shows that there are almost 2 neutral particles crossing the Faraday Cup boundary per beam proton. It is important to note that the addition of deuterons, tritons and alphas changes the gain by -0.0002\footnote{See mc-ptc-11-3.0-f for effect of deuterons, tritons, and alphas.}, and that there were no proton creation cross-sections for Ag-107, so it was substituted for Ag-109 which makes up about 2/3 of the silver.

In pursuit of the optimization of this portable Faraday Cup, we considered many variants of applicable theoretical models to corroborate with available experimental data.  Fig. \ref{fig:comp_results} depicts a very convincing similarity in behavior between the HIT beam stop measurements and the simulation of the copper Faraday Cup in air without the \emph{no} silver/Kapton layer. Shown by the thin separation between plots and occasional crossovers, there are no readily identifiable trends of increasing Kapton thickness (LATER: ONE LAST CONTROL RUN OF PURE COPPER). Replacing the air with a vacuum outermost layer increases the overall positive charge of the detector (CITE: delta ray info), however this change is quite small ($\ll 1\%$), and is therefore likely more indicitive of negligible difference\cite{bo88}.


%%%%%%%%%%%%%%%%%%%%%%%%%%%%%%%%%%%%%%%%%%%%%%%%%%%%%%%%%%%%%%%%%%%%%
\section{Conclusions}


%%%%%%%%%%%%%%%%%%%%%%%%%%%%%%%%%%%%%%%%%%%%%%%%%%%%%%%%%%%%%%%%%%%%%
\section{Acknowledgments}

We would like to express our sincerest gratitude to Paul Romano and Tom Sutton, who provided the template for this paper.


%%%%%%%%%%%%%%%%%%%%%%%%%%%%%%%%%%%%%%%%%%%%%%%%%%%%%%%%%%%%%%%%%%%%%

\setlength{\baselineskip}{12pt}
\bibliographystyle{mc2015}
\bibliography{references}

%%%%%%%%%%%%%%%%%%%%%%%%%%%%%%%%%%%%%%%%%%%%%%%%%%%%%%%%%%%%%%%%%%%%%
\appendix
\section{Experimentals Results at HIT\cite{PTC-HIT}}

\begin{table}[H]
  \centering
  \caption{Measured Gain from HIT Beam Stops}
  \begin{tabular}{lccccc}
    \toprule
    Energy (MeV) & S59 & S100 & S200 \\
    \midrule
    70.03  & \num{0.9750} & \num{0.9385} & \num{0.9350} \\
    100.46 & \num{0.9850} & \num{0.9500} & \num{0.9475} \\
    130.52 & \num{0.9925} & \num{0.9580} & \num{0.9525} \\
    160.09 & \num{1.0000} & \num{0.9635} & \num{0.9590} \\
    190.48 & \num{1.0075} & \num{0.9715} & \num{0.9650} \\
    221.06 & \num{1.0125} & \num{0.9800} & \num{0.9770} \\
    \bottomrule
  \end{tabular}
  \label{tab:HIT_data}
\end{table}

\end{document}
